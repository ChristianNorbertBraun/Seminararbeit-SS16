\newpage
\section{Rigging}
\subsection{Was ist Rigging?}
Rigging ist ein Teil der Computeranimation, wobei die Skelettanimation eines der am h�ufig genutzten Methoden ist.\\
Bei dieser Methode wird ein hierarchisch aufgebautes Skelett, ein 3D-Modell, welches im weiteren Verlauf als Mesh bezeichnet wird, und die Bewegungsdaten einzelner Punkte des Skeletts ben�tigt. Diese markanten Punkte, k�nnen zum Beispiel die Gelenke oder das Ende eines Fingers sein. Der Begriff Bewegungsdaten steht hier als Synonym f�r die 3D-Koordinaten im zeitlichen Verlauf einer Bewegung. Diese Koordinaten repr�sentieren einzelne Marker welche mit einem Motion Capturing System aufgenommen wurden. \\
Beim Rigging wird nun das Skelett mit dem Mesh verbunden. Wie sich das Mesh dann w�hrend einer Bewegung verformt, ist Teil des Skinning-Prozesses.
Rigging l�sst sich in zwei Hauptbereiche einteilen:
\begin{itemize}
	\item Manuelles Riggen
	\item Automatisches Riggen
\end{itemize}
F�r das manuelle Riggen ben�tigt man spezielle 3D Modellierungs Software wie Maya oder Blender. Ein erfahrener Fachmann muss hierbei manuell das Skelett im Mesh plazieren.\\
Diese Ausarbeitung behandelt das Thema automatisches Riggen. Auch dieser Unterpunkt des Riggings l�sst sich weiter aufteilen.
\begin{itemize}
	\item \nameref{sec:skeleton-extraction}
	\item \nameref{sec:skeleton-embedding}
\end{itemize}
Es gibt auch Ans�tze \cite{combination}, welche beide Methoden verbindet. Im weiteren Verlauf wird jedoch nur auf \nameref{sec:skeleton-extraction} und \nameref{sec:skeleton-embedding} eingegangen. Wobei der Schwerpunkt eindeutig auf \nameref{sec:skeleton-embedding} liegt.
Das Ziel des automatischen Riggings liegt darin, das der Prozess vereinfacht und beschleunigt wird. Bisher war die Plazierung des Skellets im Mesh f�r Anf�nger schwierig. Der automatisierte Rigging-Vorgang wird oft an drei Kenngr��en bemessen:

\begin{itemize}
	\item Performanz: Der Vorgang sollte nicht l�nger dauern als das manuelle Rigging.
	\item Allgemeing�ltigkeit: Der Vorgang sollte nicht auf bestimmte Modelle, z. B. humanuide Formen, begrenzt sein.
	\item Selbstst�ndigkeit: Die ben�tigte Interaktion mit dem Benutzer sollte m�glichst gering sein.
\end{itemize}
In der nachfolgenden Ausarbeitung werden die einzelnen Verfahren mithilfe dieser drei Aspekten evaluiert.\\
Um die Verwechslungsgefahr zu minimieren werden noch folgende Begriffe eingef�hrt:
\begin{itemize}
	\item Animation Skelett: Hierarchisches Skelett mit Knochen und Gelenken.
	\item Kurven Skelett: Mittelachse eines Meshs.
\end{itemize}
\newpage


\subsection{Skelett-Extraktion}
\label{sec:skeleton-extraction}

Bisschen was �ber extraktion
\subsection{Skelett-Einbindung}
\label{sec:skeleton-embedding}

Dieser Abschnitt f�hrt in einige LaTeX-Elemente ein.
\cite{embedding, extraction}
Sehr sch�n in LaTeX ist die Literaturverwaltung. Literaturzitate
einzelner \cite{Deinzer00:VSC} oder mehrere Referenzen
\cite{Graessl:02:EDD,Bronstein97,catmull74b,POVRAY:WEB} sind einfach
einzubinden. Auch das Literaturverzeichnis ist mit bibtex immer
konsistent.

\begin{figure}[t]
\includegraphics[width=5cm]{smiley.png}
\caption{Bildunterschrift f�r den Smiley. Die Nummerierung erfoglt automatisch}
\label{bildref}
\end{figure}

Bilder werden von LaTeX automatisch an geeignete Positionen
gesetzt. Diese k�nnen nat�rlich im Text referenziert werden: Der
Smiley ist in Bild~\ref{bildref}. Tabellen k�nnen auch referenziert
werden: Die zwei Musterstudenten sind in Tabelle~\ref{tabref} aufgef�hrt.

\begin{table}[t]
\begin{center}
\begin{tabular}{|l||l|l|}\hline
Matrikelnummer & Vorname & Nachname \\\hline\hline
12345 & Heinz & Mustermann \\\hline
23456 & Maria & Mstermann \\\hline
\end{tabular}
\end{center}
\caption{Tabellenunterschrift einer zentrierten Tabelle}
\label{tabref}
\end{table}

Die gro�e St�rke von LaTeX sind die M�glichkeiten Formeln zu
setzen. Dabei kann man Formeln automatisch nummerieren lassen:

\begin{equation}
\label{formel}
f_{i,j}=
\sum_{x=0}^{N-1} \sum_{y=0}^{N-1}
\frac{2 \cdot C(x)\cdot C(y)}{N} \cdot 
F_{x,y} \cdot 
\cos \left( \frac{(2i+1) \cdot x \cdot \pi}{2 \cdot N} \right) \cdot
\cos \left( \frac{(2j+1) \cdot y \cdot \pi}{2 \cdot N} \right)
\end{equation}

Formeln im Flie�text binden sich sehr einfach ein. So kann man den
bekannten Pythagoras schreiben als $a^2 + b^2 = c^2$. F�r die gleiche
Formel wie (\ref{formel}) ohne automatische Nummerierung erh�lt man,
indem man die Umgebung, in der sie sich befindet, anpasst:

\[
f_{i,j}=
\sum_{x=0}^{N-1} \sum_{y=0}^{N-1}
\frac{2 \cdot C(x)\cdot C(y)}{N} \cdot 
F_{x,y} \cdot 
\cos \left( \frac{(2i+1) \cdot x \cdot \pi}{2 \cdot N} \right) \cdot
\cos \left( \frac{(2j+1) \cdot y \cdot \pi}{2 \cdot N} \right)
\]


Abs�tze werden durch Leerzeilen abgegrenzt.

Lorem ipsum dolor sit amet, consectetuer adipiscing elit. Aenean commodo ligula 
eget dolor. Aenean massa. Cum sociis natoque penatibus et magnis dis parturient 
montes, nascetur ridiculus mus. Donec quam felis, ultricies nec, pellentesque 
eu, pretium quis, sem. Nulla consequat massa quis enim. Donec pede justo, 
fringilla vel, aliquet nec, vulputate eget, arcu. In enim justo, rhoncus ut, 
imperdiet a, venenatis vitae, justo. Nullam dictum felis eu pede mollis pretium. 
Integer tincidunt. Cras dapibus. Vivamus elementum semper nisi. Aenean vulputate 
eleifend tellus. Aenean leo ligula, porttitor eu, consequat vitae, eleifend ac, 
enim. Aliquam lorem ante, dapibus in, viverra quis, feugiat a, tellus. Phasellus 
viverra nulla ut metus varius laoreet. Quisque rutrum. Aenean imperdiet. Etiam 
ultricies nisi vel augue. Curabitur ullamcorper ultricies nisi. Nam eget dui. 
Etiam rhoncus. Maecenas tempus, tellus eget condimentum rhoncus, sem quam semper 
libero, sit amet adipiscing sem neque sed ipsum. Nam quam nunc, blandit vel, 
luctus pulvinar, hendrerit id, lorem. Maecenas nec odio et ante tincidunt 
tempus. Donec vitae sapien ut libero venenatis faucibus. Nullam quis ante. Etiam 
sit amet orci eget eros faucibus tincidunt. Duis leo. Sed fringilla mauris sit 
amet nibh. Donec sodales sagittis magna. Sed consequat, leo eget bibendum 
sodales, augue velit cursus nunc, 

Sed ut perspiciatis unde omnis iste natus error sit voluptatem accusantium 
doloremque laudantium, totam rem aperiam, eaque ipsa quae ab illo inventore 
veritatis et quasi architecto beatae vitae dicta sunt explicabo. Nemo enim ipsam 
voluptatem quia voluptas sit aspernatur aut odit aut fugit, sed quia 
consequuntur magni dolores eos qui ratione voluptatem sequi nesciunt. Neque 
porro quisquam est, qui dolorem ipsum quia dolor sit amet, consectetur, adipisci 
velit, sed quia non numquam eius modi tempora incidunt ut labore et dolore 
magnam aliquam quaerat voluptatem. Ut enim ad minima veniam, quis nostrum 
exercitationem ullam corporis suscipit laboriosam, nisi ut aliquid ex ea commodi 
consequatur? Quis autem vel eum iure reprehenderit qui in ea voluptate velit 
esse quam nihil molestiae consequatur, vel illum qui dolorem eum fugiat quo 
voluptas nulla pariatur? At vero eos et accusamus et iusto odio dignissimos 
ducimus qui blanditiis praesentium voluptatum deleniti atque corrupti quos 
dolores et quas molestias excepturi sint occaecati cupiditate non provident, 
similique sunt in culpa qui officia deserunt mollitia animi, id est laborum et 
dolorum fuga. Et harum quidem rerum facilis est et expedita distinctio. Nam 
libero tempore, cum soluta nobis est eligendi optio cumque nihil impedit quo 
minus id quod maxime placeat facere 

Li Europan lingues es membres del sam familie. Lor separat existentie es un 
myth. Por scientie, musica, sport etc, litot Europa usa li sam vocabular. Li 
lingues differe solmen in li grammatica, li pronunciation e li plu commun 
vocabules. Omnicos directe al desirabilite de un nov lingua franca: On refusa 
continuar payar custosi traductores. At solmen va esser necessi far uniform 
grammatica, pronunciation e plu sommun paroles. Ma quande lingues coalesce, li 
grammatica del resultant lingue es plu simplic e regulari quam ti del coalescent 
lingues. Li nov lingua franca va esser plu simplic e regulari quam li existent 
Europan lingues. It va esser tam simplic quam Occidental in fact, it va esser 
Occidental. A un Angleso it va semblar un simplificat Angles, quam un skeptic 
Cambridge amico dit me que Occidental es. Li Europan lingues es membres del sam 
familie. Lor separat existentie es un myth. Por scientie, musica, sport etc, 
litot Europa usa li sam vocabular. Li lingues differe solmen in li grammatica, 
li pronunciation e li plu commun vocabules. Omnicos directe al desirabilite de 
un nov lingua franca: On refusa continuar payar custosi traductores. At solmen 
va esser necessi far uniform grammatica, pronunciation e plu sommun paroles. 

Lorem ipsum dolor sit amet, consectetuer adipiscing elit. Aenean commodo ligula 
eget dolor. Aenean massa. Cum sociis natoque penatibus et magnis dis parturient 
montes, nascetur ridiculus mus. Donec quam felis, ultricies nec, pellentesque 
eu, pretium quis, sem. Nulla consequat massa quis enim. Donec pede justo, 
fringilla vel, aliquet nec, vulputate eget, arcu. In enim justo, rhoncus ut, 
imperdiet a, venenatis vitae, justo. Nullam dictum felis eu pede mollis pretium. 
Integer tincidunt. Cras dapibus. Vivamus elementum semper nisi. Aenean vulputate 
eleifend tellus. Aenean leo ligula, porttitor eu, consequat vitae, eleifend ac, 
enim. Aliquam lorem ante, dapibus in, viverra quis, feugiat a, tellus. Phasellus 
viverra nulla ut metus varius laoreet. Quisque rutrum. Aenean imperdiet. Etiam 
ultricies nisi vel augue. Curabitur ullamcorper ultricies nisi. Nam eget dui. 
Etiam rhoncus. Maecenas tempus, tellus eget condimentum rhoncus, sem quam semper 
libero, sit amet adipiscing sem neque sed ipsum. Nam quam nunc, blandit vel, 
luctus pulvinar, hendrerit id, lorem. Maecenas nec odio et ante tincidunt 
tempus. Donec vitae sapien ut libero venenatis faucibus. Nullam quis ante. Etiam 
sit amet orci eget eros faucibus tincidunt. Duis leo. Sed fringilla mauris sit 
amet nibh. Donec sodales sagittis magna. Sed consequat, leo eget bibendum 
sodales, augue velit cursus nunc, 

Sed ut perspiciatis unde omnis iste natus error sit voluptatem accusantium 
doloremque laudantium, totam rem aperiam, eaque ipsa quae ab illo inventore 
veritatis et quasi architecto beatae vitae dicta sunt explicabo. Nemo enim ipsam 
voluptatem quia voluptas sit aspernatur aut odit aut fugit, sed quia 
consequuntur magni dolores eos qui ratione voluptatem sequi nesciunt. Neque 
porro quisquam est, qui dolorem ipsum quia dolor sit amet, consectetur, adipisci 
velit, sed quia non numquam eius modi tempora incidunt ut labore et dolore 
magnam aliquam quaerat voluptatem. Ut enim ad minima veniam, quis nostrum 
exercitationem ullam corporis suscipit laboriosam, nisi ut aliquid ex ea commodi 
consequatur? Quis autem vel eum iure reprehenderit qui in ea voluptate velit 
esse quam nihil molestiae consequatur, vel illum qui dolorem eum fugiat quo 
voluptas nulla pariatur? At vero eos et accusamus et iusto odio dignissimos 
ducimus qui blanditiis praesentium voluptatum deleniti atque corrupti quos 
dolores et quas molestias excepturi sint occaecati cupiditate non provident, 
similique sunt in culpa qui officia deserunt mollitia animi, id est laborum et 
dolorum fuga. Et harum quidem rerum facilis est et expedita distinctio. Nam 
libero tempore, cum soluta nobis est eligendi optio cumque nihil impedit quo 
minus id quod maxime placeat facere 

Li Europan lingues es membres del sam familie. Lor separat existentie es un 
myth. Por scientie, musica, sport etc, litot Europa usa li sam vocabular. Li 
lingues differe solmen in li grammatica, li pronunciation e li plu commun 
vocabules. Omnicos directe al desirabilite de un nov lingua franca: On refusa 
continuar payar custosi traductores. At solmen va esser necessi far uniform 
grammatica, pronunciation e plu sommun paroles. Ma quande lingues coalesce, li 
grammatica del resultant lingue es plu simplic e regulari quam ti del coalescent 
lingues. Li nov lingua franca va esser plu simplic e regulari quam li existent 
Europan lingues. It va esser tam simplic quam Occidental in fact, it va esser 
Occidental. A un Angleso it va semblar un simplificat Angles, quam un skeptic 
Cambridge amico dit me que Occidental es. Li Europan lingues es membres del sam 
familie. Lor separat existentie es un myth. Por scientie, musica, sport etc, 
litot Europa usa li sam vocabular. Li lingues differe solmen in li grammatica, 
li pronunciation e li plu commun vocabules. Omnicos directe al desirabilite de 
un nov lingua franca: On refusa continuar payar custosi traductores. At solmen 
va esser necessi far uniform grammatica, pronunciation e plu sommun paroles. 