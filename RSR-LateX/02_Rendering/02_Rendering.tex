\section{Rendering in Filmen}
\subsection{Abstrakt}
\label{sec:renderingAbstrakt}
Rendering bezeichnet im Allgemeinen das Erstellen eines zweidimensionalen Bildes aus einer dreidimensionalen Szene oder eines Objektes unter Ber�cksichtigung physikalische Gr��en wie Licht, Schatten und Materialien. \cite{renderingDefinition}. Dieser Prozess findet in einer Vielzahl von unterschiedlichen Bereichen Anwendungsf�lle. Zum einen in der Darstellung von Computerspielen, bei Konstruktionen in CAD-Programmen aber auch bei CGI in aktuellen Filmen. 
Jeder Anwendungsfall hat andere Anforderungen und Priorit�ten an das Rendering. W�hrend Computerspiele vor Allem ein schnelles Erzeugen von Bildern ben�tigen, kommt es in Filmen auf eine m�glichst reale und detailreiche Darstellung an. Das Folgende Kapitel besch�ftigt sich mit den besonderen Anforderungen an das Rendering im Film. Hierbei wird sowohl die Erstellung komplett synthetischer Filme wie Cars oder Monster AG als auch das Einbetten synthetischer Objekte in reale Bilder betrachtet.

\subsection{Einf�hrung}


\begin{figure}[t]
\includegraphics[width=5cm]{smiley.png}
\caption{Bildunterschrift f�r den Smiley. Die Nummerierung erfoglt automatisch}
\label{bildref}
\end{figure}

Bilder werden von LaTeX automatisch an geeignete Positionen
gesetzt. Diese k�nnen nat�rlich im Text referenziert werden: Der
Smiley ist in Bild~\ref{bildref}. Tabellen k�nnen auch referenziert
werden: Die zwei Musterstudenten sind in Tabelle~\ref{tabref} aufgef�hrt.

\begin{table}[t]
\begin{center}
\begin{tabular}{|l||l|l|}\hline
Matrikelnummer & Vorname & Nachname \\\hline\hline
12345 & Heinz & Mustermann \\\hline
23456 & Maria & Mstermann \\\hline
\end{tabular}
\end{center}
\caption{Tabellenunterschrift einer zentrierten Tabelle}
\label{tabref}
\end{table}

Die gro�e St�rke von LaTeX sind die M�glichkeiten Formeln zu
setzen. Dabei kann man Formeln automatisch nummerieren lassen:

\begin{equation}
\label{formel}
f_{i,j}=
\sum_{x=0}^{N-1} \sum_{y=0}^{N-1}
\frac{2 \cdot C(x)\cdot C(y)}{N} \cdot 
F_{x,y} \cdot 
\cos \left( \frac{(2i+1) \cdot x \cdot \pi}{2 \cdot N} \right) \cdot
\cos \left( \frac{(2j+1) \cdot y \cdot \pi}{2 \cdot N} \right)
\end{equation}

Formeln im Flie�text binden sich sehr einfach ein. So kann man den
bekannten Pythagoras schreiben als $a^2 + b^2 = c^2$. F�r die gleiche
Formel wie (\ref{formel}) ohne automatische Nummerierung erh�lt man,
indem man die Umgebung, in der sie sich befindet, anpasst:

\[
f_{i,j}=
\sum_{x=0}^{N-1} \sum_{y=0}^{N-1}
\frac{2 \cdot C(x)\cdot C(y)}{N} \cdot 
F_{x,y} \cdot 
\cos \left( \frac{(2i+1) \cdot x \cdot \pi}{2 \cdot N} \right) \cdot
\cos \left( \frac{(2j+1) \cdot y \cdot \pi}{2 \cdot N} \right)
\]


Abs�tze werden durch Leerzeilen abgegrenzt.

