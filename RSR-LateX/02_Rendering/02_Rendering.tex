\section{Rendering}
\subsection{Rendering in Filmen}
\label{sec:renderingInMovies}

\cite{watt}
Dieser Abschnitt f�hrt in einige LaTeX-Elemente ein.

Sehr sch�n in LaTeX ist die Literaturverwaltung. Literaturzitate
einzelner \cite{Deinzer00:VSC} oder mehrere Referenzen
\cite{Graessl:02:EDD,Bronstein97,catmull74b,POVRAY:WEB} sind einfach
einzubinden. Auch das Literaturverzeichnis ist mit bibtex immer
konsistent.

\begin{figure}[t]
\includegraphics[width=5cm]{smiley.png}
\caption{Bildunterschrift f�r den Smiley. Die Nummerierung erfoglt automatisch}
\label{bildref}
\end{figure}

Bilder werden von LaTeX automatisch an geeignete Positionen
gesetzt. Diese k�nnen nat�rlich im Text referenziert werden: Der
Smiley ist in Bild~\ref{bildref}. Tabellen k�nnen auch referenziert
werden: Die zwei Musterstudenten sind in Tabelle~\ref{tabref} aufgef�hrt.

\begin{table}[t]
\begin{center}
\begin{tabular}{|l||l|l|}\hline
Matrikelnummer & Vorname & Nachname \\\hline\hline
12345 & Heinz & Mustermann \\\hline
23456 & Maria & Mstermann \\\hline
\end{tabular}
\end{center}
\caption{Tabellenunterschrift einer zentrierten Tabelle}
\label{tabref}
\end{table}

Die gro�e St�rke von LaTeX sind die M�glichkeiten Formeln zu
setzen. Dabei kann man Formeln automatisch nummerieren lassen:

\begin{equation}
\label{formel}
f_{i,j}=
\sum_{x=0}^{N-1} \sum_{y=0}^{N-1}
\frac{2 \cdot C(x)\cdot C(y)}{N} \cdot 
F_{x,y} \cdot 
\cos \left( \frac{(2i+1) \cdot x \cdot \pi}{2 \cdot N} \right) \cdot
\cos \left( \frac{(2j+1) \cdot y \cdot \pi}{2 \cdot N} \right)
\end{equation}

Formeln im Flie�text binden sich sehr einfach ein. So kann man den
bekannten Pythagoras schreiben als $a^2 + b^2 = c^2$. F�r die gleiche
Formel wie (\ref{formel}) ohne automatische Nummerierung erh�lt man,
indem man die Umgebung, in der sie sich befindet, anpasst:

\[
f_{i,j}=
\sum_{x=0}^{N-1} \sum_{y=0}^{N-1}
\frac{2 \cdot C(x)\cdot C(y)}{N} \cdot 
F_{x,y} \cdot 
\cos \left( \frac{(2i+1) \cdot x \cdot \pi}{2 \cdot N} \right) \cdot
\cos \left( \frac{(2j+1) \cdot y \cdot \pi}{2 \cdot N} \right)
\]


Abs�tze werden durch Leerzeilen abgegrenzt.

