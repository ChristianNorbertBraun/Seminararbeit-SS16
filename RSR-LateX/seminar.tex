\documentclass[a4paper]{article}

\usepackage{ngerman}
\usepackage{a4wide}
\usepackage{graphicx}
\usepackage[latin1]{inputenc}
\usepackage{amscd}
\usepackage{amssymb}
\usepackage{amsmath}
\usepackage{bibgerm}

%%%
%%% Style-Definition des Literaturverzeichnis
%%%
%%\bibliographystyle{alpha}
\bibliographystyle{geralpha}
 
%%\setlength{\parindent}{0pt} %kein Einzug beim Absatzbegin
\setlength{\parskip}\medskipamount %Abstand zwischen 2 Abs�tzen


\begin{document}

\setcounter{tocdepth}{3}

\pagestyle{plain}
\pagenumbering{arabic}
\sloppy

%%%
%%% Anpassung Titel und andere Angaben
%%%
\title{Rigging - Skinning - Rendering}
\author{Julian Kurz, Marcel Gro�, Christian Braun \\
Schwerpunktseminar Medieninformatik\\
Fachhochschule W�rzburg-Schweinfurt}
\date{1. Januar 2009}

\maketitle              % Erzeuge Titel

\tableofcontents

%%%
%%% Einzelne Kapitel werden so eingebunden
%%%
\section{Rigging}
\subsection{Was ist Rigging?}
{ %�ffnende Klammer f�r hintergrundbild lokal
	\usebackgroundtemplate{\includegraphics[width=\paperwidth,height=\paperheight]{00_Rigging/Pictures/rig1.jpg}}
	\begin{frame}{
			\begin{figure}
				\colorbox{black!10}{\Huge{Was ist Rigging?}}
			\end{figure}
			}
	\blfootnote{\colorbox{black!10}{Bildquelle: \cite{rig1}}}

	\pdfnote{Frage an alle}
	\pdfnote{1. Ansatz: Motion-Capture->Punktwolke->Skelett->Kein Modell}
	\pdfnote{Problem: Was ist welcher Punkt}
		
	\end{frame}
} %schlie�ende klammer f�r hintergrundbild lokal
	
{
	\usebackgroundtemplate{\includegraphics[width=\paperwidth,height=\paperheight]{00_Rigging/Pictures/rig2.jpg}}
	\begin{frame}{}
		
		\blfootnote{\colorbox{black!10}{Bildquelle: \cite{rig2}}}
		
		\pdfnote{Hierarchisch aufgebautes Skelett + 3D-Modell + Bewegungdaten}
		
	\end{frame}
}

{
	\usebackgroundtemplate{\includegraphics[width=\paperwidth,height=\paperheight]{00_Rigging/Pictures/rig3.png}}
	\begin{frame}{
			\begin{figure}
				\colorbox{black!10}{\Huge{Manuelles Riggen}}
			\end{figure}
		}
		
		\onslide<2>{
			
			\begin{figure}
				\centering
				\begin{minipage}{.4\paperwidth}
					\centering
					\includegraphics[width=.3\paperwidth]{00_Rigging/Pictures/maya.jpg}
				\end{minipage}%
				\begin{minipage}{.4\paperwidth}
					\centering
					\includegraphics[width=.3\paperwidth]{00_Rigging/Pictures/blender-plain.png}
				\end{minipage}
			\end{figure}
		}
		
		\blfootnote{\colorbox{black!10}{Bildquellen: \cite{rig3, rig4, rig5}}}
		
		\pdfnote{Erfahrener Fachmann, manuelles Platzieren des Skeletts}
		\end{frame}
}
	
	\begin{frame}{
			\begin{figure}
				\Huge{Automatisches Riggen}
			\end{figure}}
		
		\begin{itemize}
			\item Skelett-Extraktion
			\item Skelet-Einbindung
			\item \textcolor{black!60}{Kombination aus beidem}
		\end{itemize}
			
	\end{frame}
	
{
	\usebackgroundtemplate{\includegraphics[width=\paperwidth,height=\paperheight]{00_Rigging/Pictures/rig4.jpg}}
	\begin{frame}{ \colorbox{black!10}{\Huge{Performanz}}}
			
		\blfootnote{\colorbox{black!10}{Bildquelle: \cite{rig6}}}
			
		\pdfnote{Schneller als manuelles Riggen}
	\end{frame}
}

{
	\usebackgroundtemplate{\includegraphics[width=\paperwidth,height=\paperheight]{00_Rigging/Pictures/rig5.jpg}}
	\begin{frame}{\colorbox{black!10}{\Huge{Allgemeing�ltigkeit}}
			}
			
		\blfootnote{\colorbox{black!10}{Bildquelle: \cite{rig7}}}
				
		\pdfnote{Nicht begrenzt auf best. Modelle}
	\end{frame}
}

{
	\usebackgroundtemplate{\includegraphics[width=\paperwidth,height=\paperheight]{00_Rigging/Pictures/rig6.jpg}}
	\begin{frame}{\colorbox{black!10}{\Huge{Selbst�ndigkeit}}}
			
		\blfootnote{\colorbox{black!10}{Bild nachempfunden von: \cite{rig8}}}	
			
		\pdfnote{Geringe Interaktion mit dem Benutzer}
	\end{frame}
}
\subsection{Skelett-Extraktion}
\begin{frame}{
	\begin{figure}
		\Huge{Skelett-Extraktion}
	\end{figure}}
	
	\begin{itemize}
		\item Finden der optimalen Projektion
		\item Globale Suche
		\item Lokale Suche
		\item Kurven-Skelett-Extraktion
		\item Animations-Skelett-Erzeugung
	\end{itemize}
	
	\pdfnote{Kein Skelett im Voraus ben�tigt}
	\pdfnote{Einf�hrung einer geometrischen Form: 3D-Silhouette}
		
\end{frame}

\begin{frame}{
	\begin{figure}
		\Huge{Finden der optimalen Projektion}
	\end{figure}}
	
		\begin{itemize}
			\item Startpunkt: Punkt mit gr��tem $y$-Wert
			\item Suche der Nachbarn im Radius $r$
				\begin{itemize}
					\item $r$: max. euklidischer Abstand zwischen Startpunkt und seinem Nachbarn
					\item $r= \max \Vert q_{i}(x,y,z) - q(x,y,z)\Vert$
				\end{itemize}
		\end{itemize}
		
\end{frame}

\begin{frame}{
		\begin{figure}
			\Huge{Globale Suche I}
		\end{figure}}
		
		
		$$C\{c_{i}(x,y,z) \vert c_{i}(x,y,z)\in V, c'_{i}(x,y) \in C', C' \subset P, x_{c_{i}} = x_{c'_{i}}, y_{c_{i}} = y_{c'_{i}} \}$$
		
		
		\pdfnote{Wenig Verdeckung -> Max. Oberfl�sche}
		\pdfnote{C, V, P Mengen von Punkten}
		\pdfnote{C=Silhouette, V=Mesh, P=Projektion}
		
	\end{frame}

\begin{frame}{
	\begin{figure}
		\Huge{Globale Suche II}
	\end{figure}}
	
	\begin{itemize}
		\item Finden des n�chsten Punkts:
		\begin{itemize}
			\item Aufspannen des Vektors $\overrightarrow{OX}$ mit dem Startpunkt als Ursprung
			\item Antragen der Winkel von $-\pi$ bis $\pi$
			\item Punkt mit kleinsten Winkel wird Punkt $P_{2}$
		\end{itemize}
	\end{itemize}
		
\end{frame}

\begin{frame}{
	\begin{figure}
		\Huge{Globale Suche III}
	\end{figure}}
	
	\begin{itemize}
		\item Finden aller Nachbarn in $r$
		\item Suchen des Punktes mit gr��tem Winkel $\angle P_{1}P_{2}P_{3}$
		\item Globale Suche III sooft wiederholen bis Startpunkt wieder erreicht wurde
	\end{itemize}
		
\end{frame}

\begin{frame}{
	\begin{figure}
		\Huge{Globale Suche IV}
	\end{figure}}
	
	\begin{itemize}
		\item Hinzuf�gen der Tiefeninformation zu jedem Punkt der gefundenen Menge:
	\end{itemize}
	$$ C'' \{c''_{i}(x,y,z) \in C'', x_{c''} = x_{c'}, y_{c''} = y_{c'} \} $$
		
\end{frame}

\begin{frame}{
	\begin{figure}
		\Huge{Lokale Suche}
	\end{figure}}
	
	\begin{itemize}
		\item Definition des Punktes $c''_{i}(x,y,z)$ als Startpunkt $c_{i1}(x,y,z)$
		\item Suche des Punktes $c_{i2}(x,y,z)$ mit dem gr��ten Winkel $\angle c_{i2}c''_{i}c_{ij}$ 
		\item Wiederholen des letzten Schritts bis:
		\begin{itemize}
			\item der Startpunkt wieder erreicht wurde oder
			\item der Abstand zwichen den Punkten kleiner als $r$ ist
		\end{itemize}
	\end{itemize}
		
	\pdfnote{Cij ist der Vorg�nger von Ci1}
\end{frame}

{
	\usebackgroundtemplate{\includegraphics[width=\paperwidth,height=\paperheight]{00_Rigging/Pictures/rig7.png}}
	\begin{frame}{}
	\blfootnote{\colorbox{black!10}{Bildquelle: \cite{extraction}}}	
	\end{frame}
}

{
	\usebackgroundtemplate{\includegraphics[width=\paperwidth,height=\paperheight]{00_Rigging/Pictures/rig8.png}}
	\begin{frame}{\Huge{Kurven-Skelett-Extrakton I}}
		
		\blfootnote{\colorbox{black!10}{Bildquelle: \cite{extraction}}}
		
		\pdfnote{Dreiecksnetz mit Delaunay-Triangulation}
		\pdfnote{Mittelpunkt f�r Kanten die nicht Mesh ber�hren}
		\pdfnote{Mittelachse liegt nicht mittig in der Tiefe}
	\end{frame}
}

{
	\usebackgroundtemplate{\includegraphics[width=\paperwidth,height=\paperheight]{00_Rigging/Pictures/rig9.png}}
\begin{frame}{\Huge{Kurven-Skelett-Extrakton II}}
	
	\blfootnote{\colorbox{black!10}{Bildquelle: \cite{extraction}}}
		
	\pdfnote{Orthogonale Projektionen finden und ges. Vorgang wiederholen}
\end{frame}
}

{
	\usebackgroundtemplate{\includegraphics[width=\paperwidth,height=\paperheight]{00_Rigging/Pictures/rig10.png}}
	\begin{frame}{}
		
		\blfootnote{\colorbox{black!10}{Bildquelle: \cite{extraction}}}
		
		\pdfnote{Begradigen der gefundenen Mittelachsen}
		\pdfnote{Start und Ende der Achsen sind Gelenke}
		\pdfnote{Scheitelpunkte �ber 18� -> Gelenke}
	\end{frame}
}
\subsection{Skelett-Einbindung}
\begin{frame}{
	\begin{figure}
		\Huge{Skelett-Einbindung}
	\end{figure}}
	
	\begin{itemize}
		\item Volemenreduktion
		\item Graph-Erzeugung
		\item Kontinuierliche Optimierung der Skelett-Einbindung
	\end{itemize}
	
\end{frame}

\begin{frame}{
	\begin{figure}
		\Huge{Volumenreduktion I}
	\end{figure}}
	
	\begin{itemize}
		\item Finden der medialen Fl�chen
	\end{itemize}
	\begin{figure}
		\includegraphics[width=.3\textwidth]{00_Rigging/Pictures/rig11.png}
	\end{figure}
	
	\blfootnote{\colorbox{black!10}{Bildquelle: \cite{embedding}}}
	
	\pdfnote{�quivalent zur Mittelachse}
	\pdfnote{Auf diesen Fl�chen erwartet man das Skellet}
	\pdfnote{Punkt liegt auf Fl�che wenn er min. 2 Punkte mit min. Distanz zur Oberfl�sche}
\end{frame}

\begin{frame}{
	\begin{figure}
		\Huge{Volumenreduktion II}
	\end{figure}}
	
	\begin{itemize}
		\item Erstellung einer Abart der "{}Dichtesten Kugelpackung"{}
	\end{itemize}
	\begin{figure}
		\includegraphics[width=.3\textwidth]{00_Rigging/Pictures/rig12.png}
	\end{figure}
	
	\blfootnote{\colorbox{black!10}{Bildquelle: \cite{embedding}}}
	
	\pdfnote{jeder Punkt der med. Fl�che -> Mittelpunkt}
	\pdfnote{Startpunkt: Distanz: Mittelpunkt -> Oberl�che am gr��ten}
	\pdfnote{Nur hinzuf�gen wenn kein anderer Mittelpunkt umschlossen}
\end{frame}
	
\begin{frame}{
	\begin{figure}
		\Huge{Graph-Erzeugung}
	\end{figure}}
	
	\begin{figure}
		\includegraphics[width=.3\textwidth]{00_Rigging/Pictures/rig13.png}
	\end{figure}
	
	\blfootnote{\colorbox{black!10}{Bildquelle: \cite{embedding}}}
	
	\pdfnote{Jeder Mittelpunkt -> Knoten im Graph}
	\pdfnote{schneidende Kugeln oder Gabriel Nachbarn -> Kanten im Graph}
\end{frame}	

\begin{frame}{
	\begin{figure}
		\Huge{Kontinuierliche Optimierung der Skelett-Einbindung}
	\end{figure}}
	
	\begin{figure}
		\includegraphics[width=.2\textwidth]{00_Rigging/Pictures/rig14.png}
	\end{figure}
	
	\blfootnote{\colorbox{black!10}{Bildquelle: \cite{embedding}}}
	
	\pdfnote{Vereinfachung der 2. Grad Gelenke (Knie, Elbogen)}
	\pdfnote{Reduktion der Freiheitsgrade der Straffunktionen}
\end{frame}

\begin{frame}{
	\begin{figure}
		\Huge{Straffunktionen}
	\end{figure}}
	
	\begin{itemize}
		\item Beeinflussen Qualit�t und Selbst�ndigkeit des Verfahrens
		\item Beeinflussen das Einbinden des Skeletts
		\begin{itemize}
			\item Betrachten ungeeigneter Einbettungsm�glichkeiten
		\end{itemize}
		\item Liefern Gewichtung zur Akzeptanz der L�sung
		\begin{itemize}
			\item Gewichte sind Breiter-Rand-Klassifikatioren welche mit einer St�tzvektormaschine gelernt wurden
		\end{itemize}
	\end{itemize}
	
	$$\min_{i=1}^{n}\Gamma^{T}q_{i} - min_{i=1}^{m}\Gamma^{T}p_{i}$$
	
	\pdfnote{St�tzvektormaschine trennt gut/schlecht Punkte}
	\pdfnote{p=gut, q=schlecht, Gamma = Gewichtung}
	\pdfnote{Sollte ggn. 1 gehen}
		
\end{frame}	

\begin{frame}{
	\begin{figure}
		\Huge{Ungeeignete Einbettungsm�glichkeiten}
	\end{figure}}
	
	\begin{itemize}
		\item kurze Knochen
		\item unm�gliche Gelenkstellungen
		\item unterschiedliche Knochenl�ngen bei symmetrisch gekennzeichneten Knochen
		\item als F��e markierte Gelenke nicht in Bodenn�he
		\item unm�gliche Knochenausrichtungen
		\item Knochenstr�nge der L�nge 0
		\item Gelneke, welche im Graph nah beieinander im org. Skelett weit entfernt liegen
	\end{itemize}
		
\end{frame}	

\begin{frame}{
		\begin{figure}
			\Huge{Ablauf der Optimierung}
		\end{figure}}
		
		\begin{itemize}
			\item Schritweises hinzuf�gen der Gelenke zu einem hierarchischen Graphen
			\item Erneute Optimierung des Skeletts
			\begin{itemize}
				\item Wiederherstellung der wegreduzierten Gelenke und Proportionen
			\end{itemize}
		\end{itemize}
		
		\begin{figure}
			\includegraphics[width=.3\textwidth]{00_Rigging/Pictures/rig15.png}
		\end{figure}
		
		\blfootnote{\colorbox{black!10}{Bildquelle: \cite{embedding}}}
		
		\pdfnote{Punkt starten -> Beste partielle Einbdindung finden -> bis Skelett komplett verbunden}
		\pdfnote{erneute Optimierung mit neuen Straffunktionen}
\end{frame}	

\subsection{Gegen�berstellung}
\begin{frame}{
		\begin{figure}
			\Huge{Performanz}
		\end{figure}}

	\begin{table}
		\centering
		\begin{tabular}{  p{.4\linewidth} | p{.4\linewidth} }
			Skelett-Extraktion & Skelett-Einbindung \\\hline\hline
			Zeiten: 3.6 - 15.1 Sekunden & Zeiten: 12.6 - 77.1 Sekunden \\\hline
			Komplexit�t: $O(N)$ & Komplexit�t: $O(N^{2})$ \\\hline
			Flaschenhals: Findung der optimalen Silhouette &
		\end{tabular}
	\end{table}
	
\end{frame}

\begin{frame}{
	\begin{figure}
		\Huge{Allgemeing�ltigkeit}
	\end{figure}}
	
	\begin{table}
		\centering
		\begin{tabular}{  p{.4\linewidth} | p{.4\linewidth} }
			Skelett-Extraktion & Skelett-Einbindung \\\hline\hline
			Mesh muss aus zylindrischen Teilen bestehen & Stark eingeschr�nkt auf gegebenes Skelett \\\hline
			Probleme bei zu gro�er �berdeckung & Probleme mit Modellen mit vielen kleinen Knochen
		\end{tabular}
	\end{table}
	\pdfnote{Extraktion einer Fast -> L-Form}
	\pdfnote{Einbindung beschr�nkt auf Humanoide}	
\end{frame}

\begin{frame}{
	\begin{figure}
		\Huge{Selbstst�ndigkeit}
	\end{figure}}
	
	\begin{table}
		\centering
		\begin{tabular}{  p{.4\linewidth} | p{.4\linewidth} }
			Skelett-Extraktion & Skelett-Einbindung \\\hline\hline
			Einschr�nkung durch vorgegebenes Mesh & Pose und ggebenes Skelett m�ssen sich �hneln \\\hline
			Pose und �berdeckung sind wichtige Kriterien f�r Findung der Mittelachsen und Gelenke & Beide F��e m�ssen auf dem Boden stehen
		\end{tabular}
	\end{table}
		
\end{frame}
	
{
	\usebackgroundtemplate{\includegraphics[width=\paperwidth,height=\paperheight]{00_Rigging/Pictures/rig16.png}}
	\begin{frame}{}
		
		\blfootnote{\colorbox{black!10}{Bildquelle: \cite{embedding}}}
		
		\pdfnote{Probleme bei 7, 10 und 13}
		
	\end{frame}
}
\section{Skinning}
\subsection{Was ist Skinning?}
{ %öffnende Klammer für hintergrundbild lokal
	\usebackgroundtemplate{\includegraphics[width=\paperwidth,height=\paperheight]{01_Skinning/Pictures/SkinningIntro.PNG}}
	\begin{frame}{\Huge{Was ist Skinning?}}
		\blfootnote{Bildquelle: \cite{skinningcourse:2014}}
		
		\pdfnote{Frage an alle: Was soll jetzt mit dem geriggten Modell passieren}
		\pdfnote{Frage an alle: Welcher Ansatz wäre hier der Intuitive}

		
	\end{frame}
} %schließende klammer für hintergrundbild lokal

{

	\begin{frame}{\Huge{Transformationen}}

			
			\begin{figure}
				\includegraphics[width=1.0\textwidth]{01_Skinning/Pictures/Transformation.PNG}
			\end{figure}
		
		\blfootnote{Bildquelle: \cite{skinningcourse:2014}}
		
		\pdfnote{Offene Frage: Was heisst Transformation genau, wie kann man sie darstellen, was ist besonders und wichtig daran?}
		
	\end{frame}
}

{
	
	\begin{frame}{\Huge{Translation}}
		
		\begin{figure}
			\includegraphics[width=0.3\textwidth]{01_Skinning/Pictures/Translation.PNG}
		\end{figure}
		
		
		\pdfnote{multiplikation zeigen}
		
	\end{frame}
}


{
	
	\begin{frame}{\Huge{Scale}}
		
		\begin{figure}
			\includegraphics[width=0.3\textwidth]{01_Skinning/Pictures/scale.PNG}
		\end{figure}
		
		
		
		\pdfnote{multiplikation zeigen}
		
	\end{frame}
}

{
	
	\begin{frame}{\Huge{Rotation}}
		
		\begin{figure}
			\includegraphics[width=0.6\textwidth]{01_Skinning/Pictures/rotation.PNG}
		\end{figure}
		
		
		\pdfnote{multiplikation zeigen}
		
	\end{frame}
}

{ %öffnende Klammer für hintergrundbild lokal
	\usebackgroundtemplate{\includegraphics[width=\paperwidth,height=\paperheight]{01_Skinning/Pictures/physiologisch.jpg}}
	\begin{frame}{\Huge{Anatomische Modellierung}}
		\blfootnote{\colorbox{black!10}{Bildquelle: \cite{muskeln}}}
		
		\pdfnote{Aufbau des Menschen abbilden -> Daraus ergibt sich Realität}
		
		
		
	\end{frame}
} %schließende klammer für hintergrundbild lokal

{
	
	\begin{frame}{\Huge{Motion Capture}}
		
		\begin{figure}
			\includegraphics[width=0.4\textwidth]{01_Skinning/Pictures/breathing.PNG}
		\end{figure}
		
			\blfootnote{Bildquelle: \cite{dilorenzo2008breathing}}
		
		\pdfnote{Beobachting menschlicher Funktion}
		
	\end{frame}
}

	\begin{frame}{\Huge{Multipler Input}}

		
		\begin{figure}
			\includegraphics[width=0.5\textwidth]{01_Skinning/Pictures/multi.jpg}
		\end{figure}
		
		\blfootnote{Bildquelle: \cite{multi}}
		
		\pdfnote{Viel Input}
		
	\end{frame}


{ %öffnende Klammer für hintergrundbild lokal
	\usebackgroundtemplate{\includegraphics[width=\paperwidth,height=\paperheight]{01_Skinning/Pictures/geometrisch.jpg}}
	\begin{frame}{\colorbox{black!10}{\Huge{Geometrisches Skinning}}}
		\blfootnote{\colorbox{black!10}{Bildquelle: \cite{geo}}}
		
		\pdfnote{Knoten an Knochen}
		
		
		
	\end{frame}
} %schließende klammer für hintergrundbild lokal
<<<<<<< HEAD
\subsection{Lineares Modell}
	\begin{frame}{
			\begin{figure}
				\colorbox{black!10}{\Huge{Linear Skinning}}
			\end{figure}
		}
=======

	\begin{frame}{\Huge{Linear Skinning}}
>>>>>>> 145af1e69f0528f9d8f92da79bff863dc65e39d3
		
		$$v'=Wi(v)$$
		
		\pdfnote{Lineares Modell erklären}
		
	\end{frame}

{ %öffnende Klammer für hintergrundbild lokal
	\usebackgroundtemplate{\includegraphics[width=\paperwidth,height=\paperheight]{01_Skinning/Pictures/weight.png}}
	\begin{frame}{\Huge{Weights}}
		\blfootnote{\colorbox{black!10}{Bildquelle: \cite{weights}}}
		
		\pdfnote{Abhängigkeit vom Knochen}
		
		
		
	\end{frame}
} %schließende klammer für hintergrundbild lokal

	\begin{frame}{\Huge{Heat Equilibrium}}
		
		\begin{figure}
			\includegraphics[width=0.6\textwidth]{01_Skinning/Pictures/heat.png}
		\end{figure}
		
		\blfootnote{Bildquelle: \cite{heat}}
		
		\pdfnote{Aufteilung anhand von Hitze}
		
	\end{frame}
	
	\begin{frame}{\Huge{Linear Blend Skinning}}
		
		$$v'=\sum_{i=1}^{N}wiWji(v)$$
		
		\pdfnote{Lineares Blend Skinning Modell erklären}
		
	\end{frame}
	
		\begin{frame}{\Huge{Linear Blend Skinning Matrix}}
			
			\begin{figure}
				\includegraphics[width=1.0\textwidth]{01_Skinning/Pictures/lbsmatrix.png}
			\end{figure}
			
			\blfootnote{Bildquelle: \cite{skinningcourse:2014}}
			
			\pdfnote{LBS Matrix erklären}
			
		\end{frame}
		
		\begin{frame}{\Huge{Linear Blend Skinning Artefakte}}
			
			\begin{figure}
				\includegraphics[width=1.2\textwidth]{01_Skinning/Pictures/rotationsartefakt.png}
			\end{figure}
			
			\blfootnote{Bildquelle: \cite{weights}}
			
			\pdfnote{Aufteilung anhand von Hitze}
			
		\end{frame}
		
				\begin{frame}{\Huge{Linear Blend Skinning Artefakte}}
					
					\begin{figure}
						\includegraphics[width=0.6\textwidth]{01_Skinning/Pictures/interpolation_angle.png}
					\end{figure}
					
					\blfootnote{Bildquelle: \cite{weights}}
					
					\pdfnote{Aufteilung anhand von Hitze}
					
				\end{frame}

	\begin{frame}{\Huge{Facial Blend Skinning}}

		
		$$S=S0+\sum_{i}wi(Si-S0)$$
		
		\pdfnote{Lineares Blend Skinning Modell erklären}
		
	\end{frame}

				\begin{frame}{\Huge{Facial Blend Skinning}}
					
					\begin{figure}
						\includegraphics[width=1.2\textwidth]{01_Skinning/Pictures/face.png}
					\end{figure}
					
					\blfootnote{Bildquelle: \cite{dutreve2008feature}}
					
					\pdfnote{Aufteilung anhand von Hitze}
					
				\end{frame}
<<<<<<< HEAD
\subsection{Quaternionen}
				\begin{frame}{
						\begin{figure}
							\colorbox{black!10}{\Huge{Quaternionen}}
						\end{figure}
					}
=======

				\begin{frame}{\Huge{Quaternionen}}
>>>>>>> 145af1e69f0528f9d8f92da79bff863dc65e39d3
					
					\begin{figure}
						\includegraphics[width=0.5\textwidth]{01_Skinning/Pictures/Quaternionen.png}
					\end{figure}
					
					\blfootnote{Bildquelle: \cite{kavan2008geometric}}
					
					\pdfnote{Video 1}
					
				\end{frame}
				
		\begin{frame}{\Huge{Dual Quaternionen Skinning}}
			
			$$\mathbf{\dot q} =  \frac{\sum_{i=1}^n w_i \mathbf{\dot q}_i}{\| \sum_{i=1}^n w_i \mathbf{\dot q}_i \|}$$
			
			\pdfnote{Lineares Blend Skinning Modell erklären}
			
		\end{frame}
		
				\begin{frame}{\Huge{Linear Blend Skinning}}
					
					\begin{figure}
						\includegraphics[width=1.0\textwidth]{01_Skinning/Pictures/lbsgrad.png}
					\end{figure}
					
					\blfootnote{Bildquelle: \cite{weights}}
					
					\pdfnote{Aufteilung anhand von Hitze}
					
				\end{frame}
				
						\begin{frame}{\Huge{Dual Quaternion Skinning}}
							
							\begin{figure}
								\includegraphics[width=1.0\textwidth]{01_Skinning/Pictures/dqsgrad.png}
							\end{figure}
							
							\blfootnote{Bildquelle: \cite{weights}}
							
							\pdfnote{Aufteilung anhand von Hitze}
							
						\end{frame}
						
				\begin{frame}{\Huge{Artefakte Dual Quaternionen}}
					
					\begin{figure}
						\includegraphics[width=1.0\textwidth]{01_Skinning/Pictures/dqsgradartefakte.png}
					\end{figure}
					
					\blfootnote{Bildquelle:\cite{kavan2008geometric}}
					
					\pdfnote{Video 1}
					
				\end{frame}
				
					\begin{frame}{\Huge{Fazit}}
							
							\begin{figure}
								\includegraphics[width=0.8\textwidth]{01_Skinning/Pictures/lbsvsdqs.png}
							\end{figure}
							
							\blfootnote{Bildquelle: \cite{weights}}
							
							\pdfnote{Zusammenfassung, video 2}
							
<<<<<<< HEAD
						\end{frame}
		
		
	
=======
						\end{frame}
>>>>>>> 145af1e69f0528f9d8f92da79bff863dc65e39d3

\section{Rendering}
	
	\begin{frame}[plain, c]
		
	\begin{center}
		\Huge Rendering
	\end{center}
	
	\begin{figure}
				\centering
				\begin{minipage}{.4\paperwidth}
					\centering
					\includegraphics[width=.3\paperwidth]{02_Rendering/img/triangulars.png}
				\end{minipage}%
				\begin{minipage}{.4\paperwidth}
					\centering
					\includegraphics[width=.4\paperwidth]{02_Rendering/img/rendered.png}
				\end{minipage}
		\end{figure}

		
	\end{frame}
	\pdfnote{Whatever}
	\pdfnote{zrdz}
	\pdfnote{zrdz}
	\pdfnote{zrdz}
	\pdfnote{zrdz}
	

		
		% Commands to include a figure:
		%\begin{figure}
		%\includegraphics[width=\textwidth]{your-figure's-file-name}
		%\caption{\label{fig:your-figure}Caption goes here.}
		%\end{figure}
		
	
	\begin{frame}{\Huge{Rendering in der Praxis}}
		
		\begin{figure}
				\centering
				\begin{minipage}{.4\paperwidth}
					\centering
					\includegraphics[width=.4\paperwidth]{02_Rendering/img/journey.jpg}
				\end{minipage}%
				\begin{minipage}{.5\paperwidth}
					\centering
					\includegraphics[width=.4\paperwidth]{02_Rendering/img/cad.jpg}
				\end{minipage}
		\end{figure}
		
		\blfootnote{Bildquellen: \cite{journey, cad}}
		
\end{frame}
	
\begin{frame}{\Huge{Rendering im Film}}
		
		\begin{figure}
				\centering
					\includegraphics[width=.8\paperwidth]{02_Rendering/img/cars.png}
				\centering
		\end{figure}
		
		\begin{itemize}
			\item Komplexe Szenen
			\item Viele Lichtquellen
			\item 24 Bilder pro Sekunde
		\end{itemize}
		
		\blfootnote{Bildquellen: \cite{cars}}
	\end{frame}
	
	\subsection{Licht}
	
\begin{frame}{\Huge{Licht}}
		
		\begin{figure}
				\centering
					\includegraphics[width=.8\paperwidth]{02_Rendering/img/reflexion.png}
				\centering
		\end{figure}
		
		\begin{itemize}
			\item Ideale spekulare Reflexion
			\item Reale spekulare Reflexion
			\item Diffuse Reflexion
		\end{itemize}
		
		\blfootnote{Bildquellen: \cite{reflexion}}
	\end{frame}
	
	\begin{frame}{\colorbox{black!10}{\Huge{Materialien}}}
		
		\begin{figure}
				\centering
					\includegraphics[width=.8\paperwidth]{02_Rendering/img/materials.png}
				\centering
		\end{figure}
		
		\blfootnote{Bildquellen: \cite{realImages}}
	\end{frame}
	
\begin{frame}{\Huge{Lokale Reflexionsmodelle}}
		
		  \begin{tabular}{cl}  
         \begin{tabular}{c}
           \includegraphics[height=3cm, width=3cm]{02_Rendering/img/directLight.png}
           \end{tabular}
           & \begin{tabular}{l}
             \parbox{0.6\linewidth}{%  change the parbox width as appropiate
             \textbf{Bi-Directional Reflection Distribution Function(BRDF)}
             

$$\mathrm{BRDF}=f(\theta_\mathrm{in}, \phi_\mathrm{in}, \theta_\mathrm{ref}, \phi_\mathrm{ref}) = f(\vec{\omega_{i}}, \vec{\omega_{o}})$$
   
    }
         \end{tabular}  \\
\end{tabular}
		
		\blfootnote{Bildquellen: \cite{directLight}}
\end{frame}
	
\begin{frame}{\Huge{Lambertsche Reflexion}}
		
		  \begin{tabular}{cl}  
         \begin{tabular}{c}
           \includegraphics[width=5cm]{02_Rendering/img/lambertian.jpg}
           \end{tabular}
           & \begin{tabular}{l}
             \parbox{0.4\linewidth}{%  change the parbox width as appropiate            

$$I_{D} =\vec{\omega_{i}} \cdot \vec{n}CI_{L}$$


   
    }
         \end{tabular}  \\
\end{tabular}
\end{frame}

\begin{frame}{\Huge{Blinn-Phong BRDF}}
		
		\begin{figure}
				\centering
				\begin{minipage}{.4\paperwidth}
					\centering
					\includegraphics[width=.4\paperwidth]{02_Rendering/img/glossy.jpg}
				\end{minipage}%
				\begin{minipage}{.5\paperwidth}
					\centering
					\includegraphics[width=.4\paperwidth]{02_Rendering/img/h.png}
				\end{minipage}
		\end{figure}
		
    

$$f_{s}(\vec{\omega_{i}}, \vec{\omega_{o}})=\frac{k_{L}}{\pi} + k_{G}\frac{8 + s}{8\pi}z^s$$  $$\mathrm{\quad mit\quad}z=\max(0,\vec{h} \cdot \vec{n})$$
$$\mathrm{\quad mit\quad}\vec{h}=\frac{\vec{\omega_{i}} + \vec{\omega_o}}{2}$$

		\blfootnote{Bildquellen: \cite{h}}  

\end{frame}

\begin{frame}{\Huge{Globale Beleuchtungsmodelle}}
		
		\begin{figure}
				\centering
					\includegraphics[width=.7\paperwidth]{02_Rendering/img/cornellbox.jpg}
				\centering
		\end{figure}

		\blfootnote{Bildquellen: \cite{cornell}}  

\end{frame}

\begin{frame}{\Huge{Die Rendering-Gleichung}}
	
	\begin{center}
		$$L(x, x')=g(x, x') \cdot \lbrack L_e(x,x') + \int_{s} p(x,x',x'')L(x',x'')dx''\rbrack$$
	\end{center}
	
	\begin{itemize}
		\item Analytisch nicht berechenbar
		\item Ann�herung durch Raytracer, Radiosity, etc.
	\end{itemize}
	
\end{frame}

\subsection{Rendering-Techniken}
\begin{frame}{\Huge{Rendering-Techniken}}
	
	\begin{itemize}
		\item Ann�herung an Rendering-Gleichung
		\item Modellkomplexit�t
		\item Model Diversit�t
		\item Komplexes Shading
		\item Geschwindigkeit
		\item Bildqualit�t
		\item Flexibilit�t
	\end{itemize}
	
\end{frame}

\subsubsection{Reyes-Bild-Rendering-Architektur}
\begin{frame}{\Huge{REYES}}
	
	\begin{block}{Problem}
		Komplexe Szenen sind zu gro� um sie komplett im Hauptspeicher zu halten.
	\end{block}
	
	\vskip 1cm
	
	\begin{block}{L�sung}
		Prinzip der geometrischen Lokalit�t
	\end{block}
\end{frame}
\pdfnote{Render Everything you ever saw }

\begin{frame}{\Huge{Maps}}
		\begin{figure}
			\centering
			\begin{minipage}{.4\paperwidth}
				\centering
				\includegraphics[width=.4\paperwidth]{02_Rendering/img/texture.png}
			\end{minipage}%
			\begin{minipage}{.4\paperwidth}
				\centering
				\includegraphics[width=.4\paperwidth]{02_Rendering/img/textured.png}
			\end{minipage}
		\end{figure}
		\begin{figure}
			\begin{minipage}{.4\paperwidth}
				\centering
				\includegraphics[width=.4\paperwidth]{02_Rendering/img/environment.png}
			\end{minipage}
			\begin{minipage}{.4\paperwidth}
				\centering
				\includegraphics[width=.4\paperwidth]{02_Rendering/img/environmented.png}
			\end{minipage}
		\end{figure}
		\blfootnote{Bildquellen: \cite{maps}} 

\end{frame}

\begin{frame}{\Huge{Algorithmus}}

\begin{tabular}{cl}  
	\begin{tabular}{c}
		\includegraphics[width=4.5cm]{02_Rendering/img/reyesalgo.png}
	\end{tabular}
	& \begin{tabular}{l}
		\parbox{0.4\linewidth}{%  change the parbox width as appropiate            
			
		\begin{itemize}
			\item bounding
			\item dicing
			\item splitting
		\end{itemize}
			
		}
	\end{tabular}  \\
\end{tabular}
\blfootnote{Bildquellen: \cite{REYES}} 
\end{frame}

\begin{frame}{\Huge{Algorithmus}}
		\begin{figure}
			\centering
			\begin{minipage}{.4\paperwidth}
				\centering
				\includegraphics[width=.4\paperwidth]{02_Rendering/img/reyesvisible.png}
			\end{minipage}%
			\begin{minipage}{.5\paperwidth}
				\centering
				\includegraphics[width=.4\paperwidth]{02_Rendering/img/sphere.png}
			\end{minipage}
		\end{figure}

	\blfootnote{Bildquellen: \cite{REYES}} 
\end{frame}

\begin{frame}{\Huge{Beispiele}}
	\begin{figure}
		\centering
		\begin{minipage}{.4\paperwidth}
			\centering
			\includegraphics[width=.4\paperwidth]{02_Rendering/img/lamp.png}
		\end{minipage}%
		\begin{minipage}{.5\paperwidth}
			\centering
			\includegraphics[width=.4\paperwidth]{02_Rendering/img/monsterag.png}
		\end{minipage}
	\end{figure}
\blfootnote{Bildquellen: \cite{REYES, monster}} 
\end{frame}

\subsubsection{Raytracing}

\begin{frame}{\Huge{Raytracing}}

	\begin{figure}
		\centering
		\includegraphics[width=.8\paperwidth]{02_Rendering/img/BallsRender.png}
		\centering
	\end{figure}
	\blfootnote{Bildquellen: \cite{rayTracingBalls}} 
\end{frame}

\begin{frame}{\Huge{Reflexionen}}
	\begin{figure}
		\centering
		\begin{minipage}{.4\paperwidth}
			\centering	
			\includegraphics[width=.3\paperwidth]{02_Rendering/img/map.png}
		\end{minipage}%
		\begin{minipage}{.5\paperwidth}
			\centering
			\includegraphics[width=.3\paperwidth]{02_Rendering/img/traced.png}
		\end{minipage}
	\end{figure}
	
		\begin{block}{Vorteile}
			\begin{itemize}
				\item Realistische Reflexionen
				\item Abbildung beliebig vieler Lichtquellen
				\item Spiegelungen spekularer Objekte ineinander
			\end{itemize}
			
		\end{block}
	
	\blfootnote{Bildquellen: \cite{cars}} 
\end{frame}

\begin{frame}{\Huge{Raytracer Algorithmus}}
	

	\begin{figure}

		\begin{minipage}{.5\paperwidth}

			\includegraphics[width=.4\paperwidth]{02_Rendering/img/rayTracing.png}
		\end{minipage}%
		\begin{minipage}{.5\paperwidth}

			\includegraphics[width=.4\paperwidth]{02_Rendering/img/rayTracing2.png}
		\end{minipage}
	\end{figure}
	
	\begin{block}{Performance}
		\begin{itemize}
			\item Schnittpunkttest muss f�r Schatten und Objektberechnung durchgef�hrt werden
			\item Gesamte Szene muss im Hauptspeicher liegen
		\end{itemize}

	\end{block}

	\blfootnote{Bildquellen: \cite{rayTracing}} 
\end{frame}

\begin{frame}{\Huge{H�llk�rper}}
	
	
 \begin{tabular}{cl}  
 	\begin{tabular}{c}
 		\includegraphics[width=4cm]{02_Rendering/img/rabbit.jpg}
 	\end{tabular}
 	& \begin{tabular}{l}
 		\parbox{0.6\linewidth}{%  change the parbox width as appropiate            
 			
 		\begin{itemize}
 			\item H�llk�rper zeichnen leichte Schnittpunktberechnung aus
 			\item H�llk�rper k�nnen Gruppen von Objekten umgeben
 		\end{itemize}
 			
 			
 			
 		}
 	\end{tabular}  \\
 \end{tabular}
	
	\blfootnote{Bildquellen: \cite{rabbit}} 
\end{frame}

\begin{frame}{\Huge{Strahlenb�ndel}}
	
	
	\begin{tabular}{cl}  
		\begin{tabular}{c}
			\includegraphics[width=4.5cm]{02_Rendering/img/raydiffer.jpg}
		\end{tabular}
		& \begin{tabular}{l}
			\parbox{0.6\linewidth}{%  change the parbox width as appropiate            
				
				\begin{itemize}
					\item Nachbarstrahlen beschreiten anfangs einen �hnlichen Weg
					\item Strahlenb�ndelschnittpunkte erzeugen neue Strahlenb�ndel
				\end{itemize}
				
				
				
			}
		\end{tabular}  \\
	\end{tabular}
	
	\blfootnote{Bildquellen: \cite{raydiff}} 
\end{frame}


\begin{frame}{\Huge{Beispiele}}
	
	
	\begin{figure}
		\centering
		\begin{minipage}{.5\paperwidth}
			
			\includegraphics[width=.4\paperwidth]{02_Rendering/img/raytracerCars.png}
		\end{minipage}%
		\begin{minipage}{.5\paperwidth}
			\includegraphics[width=.35\paperwidth]{02_Rendering/img/firstSzene.jpg}
		\end{minipage}
	\end{figure}
	
	\blfootnote{Bildquellen: \cite{cars, firstSzene}} 
\end{frame}

\subsection{Einf�gen eines synthetischen Models in eine reale Szene}
\begin{frame}{\Huge{Synthetisches Model + reale Szene}}
	
	
	\begin{figure}
		\centering
		\includegraphics[width=.7\paperwidth]{02_Rendering/img/gollum.jpg}
	\end{figure}
	
	\blfootnote{Bildquellen: \cite{gollum}} 
\end{frame}

\begin{frame}{\Huge{Szenen Teile}}
	
	
	\begin{tabular}{cl}  
		\begin{tabular}{c}
			\includegraphics[width=4.5cm]{02_Rendering/img/scenes.png}
		\end{tabular}
		& \begin{tabular}{l}
			\parbox{0.6\linewidth}{%  change the parbox width as appropiate            
				
				\begin{itemize}
					\item Distanzszene
					\item Lokale Szene
					\item Synthetisches Objekt
				\end{itemize}
			}
		\end{tabular}  \\
	\end{tabular}
	
	\blfootnote{Bildquellen: \cite{realImages}} 
\end{frame}

\begin{frame}{\Huge{Lichtbasiertes Model der Distanzszene}}
	
	\begin{itemize}
		\item Unabh�ngig von anderen Teilen der Szene
		\item Hauptverantwortlich f�r Beleuchtung der Szene
		\item Ben�tigt genaue Informationen �ber die Lichtintensit�t an jedem Pixel
		\item Intensit�t wird in High Dynamic Ranger Images gespeichert
	\end{itemize}

\end{frame}

\begin{frame}{\Huge{Materialbasiertes Model der lokalen Szene}}
	
	\begin{figure}
		\centering
		\begin{minipage}{.5\paperwidth}
			
			\includegraphics[width=.4\paperwidth]{02_Rendering/img/background.png}
		\end{minipage}%
		\begin{minipage}{.5\paperwidth}
			\includegraphics[width=.35\paperwidth]{02_Rendering/img/localSzeneOnly.png}
		\end{minipage}
	\end{figure}
	
	\begin{itemize}
		\item Empf�ngt Schatten und Reflexion von synthetischen Model
		\item Kann Licht auf synthetisches Model werfen und es verdecken
		\item Muss sowohl Beleuchtungs- als auch Materialeigenschaften vorweisen(BRDFS)
	\end{itemize}

	\blfootnote{Bildquellen: \cite{realImages}} 
	
\end{frame}

\begin{frame}{\Huge{Synthetisches Model}}
	
	\begin{figure}
		\centering
		\begin{minipage}{.5\paperwidth}
			
			\includegraphics[width=.4\paperwidth]{02_Rendering/img/model1.png}
		\end{minipage}%
		\begin{minipage}{.5\paperwidth}
			\includegraphics[width=.35\paperwidth]{02_Rendering/img/model2.png}
		\end{minipage}
	\end{figure}
	
\end{frame}

\begin{frame}{\Huge{Sonderf�lle}}
	
	\begin{itemize}
		\item Konzentriertes Licht bestrahlt das synthetische Model und dieses verursacht starke Schatten
		\item Konzentriertes Licht bestrahlt ein spiegelndes, synthetische Model und dieses wirft Licht in die Szene
		\item Spekulare Oberfl�chen der Distanzszene k�nnen vom Model beeinflusst werden
		\item Das synthetische Model strahlt Licht aus
	\end{itemize}
	
\end{frame}

\subsubsection{Beleuchtung}

\begin{frame}{\Huge{Beleuchtung}}
	
		\begin{tabular}{cl}  
			\begin{tabular}{c}
				\includegraphics[width=4.5cm]{02_Rendering/img/room.png}
			\end{tabular}
			& \begin{tabular}{l}
				\parbox{0.6\linewidth}{%  change the parbox width as appropiate            
					
					\begin{itemize}
						\item Kann manuell geschehen
						\item Intensit�ten m�ssen gesch�tzt werden
						\item Globale Beleuchtung muss modelliert werden
					\end{itemize}
				}
			\end{tabular}  \\
		\end{tabular}
\end{frame}

\begin{frame}{\Huge{Automatisierte Beleuchtung}}
	
	\begin{figure}
		\centering
		\begin{minipage}{.2\paperwidth}
			\includegraphics[width=.2\paperwidth]{02_Rendering/img/prob1.jpg}
		\end{minipage}%
		\begin{minipage}{.2\paperwidth}
			\includegraphics[width=.2\paperwidth]{02_Rendering/img/prob2.jpg}
		\end{minipage}
		\begin{minipage}{.2\paperwidth}
			\includegraphics[width=.2\paperwidth]{02_Rendering/img/propfull.jpg}
		\end{minipage}
	\end{figure}
	
	\begin{itemize}
		\item Erstellen mehrerer HDRI einer Spiegelkugel
		\item Verschmelzen der HDRIs zum erstellen einer Lichtsonde
	\end{itemize}
	
\end{frame}

\begin{frame}{\Huge{Lichtbasiertes Model erstellen }}
	
		\begin{tabular}{cl}  
			\begin{tabular}{c}
				\includegraphics[width=.4\paperwidth]{02_Rendering/img/algo3.png}
				\includegraphics[width=.4\paperwidth]{02_Rendering/img/distantszene.png}
			\end{tabular}
			& \begin{tabular}{l}
				\parbox{0.6\linewidth}{%  change the parbox width as appropiate            
					
					\begin{itemize}
						\item Kann manuell geschehen
						\item Intensit�ten m�ssen gesch�tzt werden
						\item Globale Beleuchtung muss modelliert werden
					\end{itemize}
				}
			\end{tabular}  \\
		\end{tabular}
	
	\begin{itemize}
		\item Strahlen bis Schnittpunkt mit Box verfolgen
	\end{itemize}
	
		\blfootnote{Bildquellen: \cite{realImages}} 
\end{frame}

\begin{frame}{\Huge{Ergebnis}}
	
	\begin{figure}
		\centering
		\includegraphics[width=.4\paperwidth]{02_Rendering/img/algo4.png}
	\end{figure}
	
	\begin{itemize}
		\item Raytracing zum rendern der Szene
		\item Strahlen vom Model �ber die lokale Szene bis in die Distanzszene verfolgen
	\end{itemize}
		\blfootnote{Bildquellen: \cite{realImages}} 
\end{frame}

\begin{frame}{\Huge{Einf�gen des synthetischen Models}}
		\begin{figure}
			\centering
			\begin{minipage}{.4\paperwidth}
				\centering
				\includegraphics[width=.4\paperwidth]{02_Rendering/img/background.png}
			\end{minipage}%
			\begin{minipage}{.4\paperwidth}
				\centering
				\includegraphics[width=.4\paperwidth]{02_Rendering/img/calibration.png}
			\end{minipage}
		\end{figure}
		\begin{figure}
			\begin{minipage}{.4\paperwidth}
				\centering
				\includegraphics[width=.4\paperwidth]{02_Rendering/img/localszene.png}
			\end{minipage}
			\begin{minipage}{.4\paperwidth}
				\centering
				\includegraphics[width=.4\paperwidth]{02_Rendering/img/final.png}
			\end{minipage}
		\end{figure}
		\blfootnote{Bildquellen: \cite{realImages}} 
\end{frame}


\begin{frame}{\Huge{Fazit}}
	
		\begin{block}{Vorteile}
			\begin{itemize}
				\item Geometrie- und Materialeigenschaften f�r lokale Szene m�ssen nur einmal bestimmt werden
				\item Lichtbasiertes Model der Distanzszene muss nur einmal bestimmt werden
				\item Beleuchtung wirkt korrekt auf synthetische Objekte ein
			\end{itemize}
		\end{block}
		
		
		\begin{block}{Nachteile}
			\begin{itemize}
				\item BRDFs f�r lokale Szene k�nnen nur gesch�tzt werden
				\item Kamera muss kalibriert werden
			\end{itemize}
		\end{block}
\end{frame}



%%%
%%% Literatur-Verzeichnis
%%%

\bibliography{seminar}

\end{document}
