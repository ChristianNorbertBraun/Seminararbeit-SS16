\documentclass[a4paper]{article}

\usepackage{ngerman}
\usepackage{a4wide}
\usepackage{graphicx}
\usepackage[latin1]{inputenc}
\usepackage{amscd}
\usepackage{amssymb}
\usepackage{amsmath}
\usepackage{bibgerm}

%%%
%%% Style-Definition des Literaturverzeichnis
%%%
%%\bibliographystyle{alpha}
\bibliographystyle{geralpha}
 
%%\setlength{\parindent}{0pt} %kein Einzug beim Absatzbegin
\setlength{\parskip}\medskipamount %Abstand zwischen 2 Abs�tzen


\begin{document}

\setcounter{tocdepth}{3}

\pagestyle{plain}
\pagenumbering{arabic}
\sloppy

%%%
%%% Anpassung Titel und andere Angaben
%%%
\title{Seminarthema}
\author{Heinz Mustermann \\
Schwerpunktseminar Medieninformatik\\
Fachhochschule W�rzburg-Schweinfurt}
\date{1. Januar 2009}

\maketitle              % Erzeuge Titel

\tableofcontents

%%%
%%% Einzelne Kapitel werden so eingebunden
%%%
\input{seminar1.tex}

%%%
%%% Literatur-Verzeichnis
%%%

\bibliography{seminar}

\end{document}
