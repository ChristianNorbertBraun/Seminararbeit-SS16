\section{Überblick}
Die folgende Ausarbeitung behandelt alle nötigen Schritte um ein synthetisches Model zu animieren und nahtlos in eine reale Filmszene einzufügen. Dieser Prozess lässt sich in die drei Bereiche Rigging, Skinning und Rendering aufteilen. Beim Rigging wird ein Animations-Skelett auf den synthetischen Charakter gemapped, welches realistische Bewegungen ermöglicht. Damit sich der Körper beim Bewegen der Gliedmaßen korrekt verformt, wendet man Skinning an. So werden zum Beispiel beim Beugen oder Strecken eines Elements die beteiligten Oberflächen gestreckt oder gestaucht. Zu guter Letzt wird das Model noch gerendert. Über das Hinzufügen von Materialeigenschaften und die Interaktion mit Licht kann das Model in komplett synthetische oder auch in reale Szenen eingefügt werden.