\subsection{Grundlegendes lineares Skinning-Modell} 

Wie zuvor erl�utert basieren die geometrischen Skinning-Verfahren darauf, passiv die Bewegungen des Skelettes mitzuverfolgen. Wenn der Gittergraph also die Bewegungen eines Knochen mitverfolgen soll, w�re der n�chstliegendste Ansatz, jedem Punkt $v$ einen Knochen $i$ zuzuweisen. Nun wird die Transformation des Knochen $Wi$ an einem Punkt vollf�hrt, um den transformierten Punkt $v'$ zu erhalten. 

\begin{equation}
\label{formel}
v'=Wi(v)
\end{equation}

Wenn sich das Skelett bewegt, bedeutet das immer, dass sich einzelne Knochen bewegen. Wir m�ssen also f�r jeden Knochen isoliert betrachten, wie er sich bewegt. Dann beobachten wir, welche Punkte ihm zugeordnet werden. Diese Punkte bewegen wir auf die selbe Weise wie den Knochen. Als Resultat wird jeder Punkt an einem Knochen, der sich bewegt, verschoben, wohingegen die Haut an Stellen, wo der Knochen sich relativ zu seinem Vater-Knochen nicht bewegt, ruhig verharrt. 

Dieser Ansatz funktioniert gut f�r feste Materialien wie Stein, Holz oder Metall, beispielsweise also bei einem Roboter. Organisches Material, wie Stoff oder Haut, verformt sich bei Bewegungen jedoch ungleichm��ig und gegeneinander. Besonders deutlich ist dies an menschlichen Gelenken zu beobachten, wo sich Haut und Muskeln gegeneinander aufbauen. Beim obigen Ansatz w�rde dann zum Beispiel im Falle einer Bewegung des Oberarms, bei der der Unterarm aber still bleibt, die Haut am Ellenbogen irgendwann pl�tzlich aufh�ren, sich mit zubewegen. Das passiert in dem Moment, wenn die betroffenen Stellen nicht mehr dem Oberarm, sondern dem Unterarm zugeordnet sind. 

Es ist trotzdem sinnvoll, sich diesen einfachen Ansatz zu merken, weil jede Abweichung davon in der Struktur unseres Modells begr�ndet ist und jeder abweichende Ansatz sich bei festem Material wieder auf diesen reduzieren lassen muss.