\paragraph{Skinning im Gesicht}

Wie in den meisten Animationen ist es auch unser Ziel, dass die Betrachter auf einem emotionalen Level Bezug zu unserem Charakter aufnehmen. Daf�r ist das Gesicht ein wichtiger Faktor. Die verschiedenen Gesichtsausdr�cke sind jedoch wenig abh�ngig vom anatomischen Skelett. Wir k�nnen die Situation durch einsetzten von Pseudoknochen verbessern, aber nicht l�sen.

Nach /cite{dutreve2008feature} ist dieser interessante Ansatz entstanden, der auf "{}Blend-Shaping"{} basiert, auch wenn der Ansatz zugegebenerma�en etwas mehr Eingriff von Seiten des K�nstlers bedarf. Die grundlegende Idee ist einfach: Wir erstellen eine neutrale Gesichtsposition $S0$ und legen ben�tigte Emotionen fest, zum Beispiel Freude oder Stolz. F�r jede dieser Emotionen erzeugen wir ein Beispiel-Gesicht ($S1$,$S2$,...). Nun erstellen wir glaubw�rdige Gesichtspositionen, indem wir die Emotionen mischen und als Gewicht heranziehen. Wollen wir zum Beispiel Stolz und Freude zu gleichen Teilen, w�re dies $0.5*S1+0.5S2$. Man beachte, dass bei diesem Ansatz der K�nstler grundlegend mehr Freiheiten hat und die Gewichte zum Einen in ihrer Summe 1 sein m�ssen, zum Anderen k�nnen wir hier auch mit negativen Gewichten arbeiten. Am Ende erhalten wir folgende Formel:

\begin{equation}
\label{formel}
S=S0+\sum_{i}wi(Si-S0)
\end{equation}
