\subsection{Skinning im Gesicht}

Wie bei den meisten ist auch bei uns das Ziel das die Betrachter auf einem emotionalen Level Bezug zu unserem Charakter aufnehmen. Dafür ist jedoch bekannterweise das Gesicht ein wichtiger Faktor. Leider sind die verschiedenen Gesichtsausdrücke wenig abhängig von unserem anatomischen Skelett. Wir können die Situation durch einsetzten von Pseudoknochen verbessern, aber nicht lösen.

Nach (Quelle) ist aber dieser interessante Ansatz entstanden der auf "Blend-Shaping" basiert. Auch wenn dieser zugegebenermaßen etwas mehr Eingriff von Seiten des Künstlers bedarf. Die Grundlegende Idee ist einfach. Wir erstellen eine neutrale Gesichtsposition $S0$ und überlegen uns Emotionen welche wir wollen, zum Beispiel Freude oder Stolz. Für jede dieser Emotionen erzeugen wir ein Beispiel Gesicht ($S1$,$S2$,...). Nun erstellen wir glaubwürdige Gesichtspositionen indem wir die Emotionen mischen und als Gewicht nehmen. Wollen wir zum Beispiel Stolz und Freude zu gleichen teilen wäre dies $0.5*S1+0.5S2$. Man beachte das bei diesem Ansatz der Künstler grundlegend mehr Freiheiten hat und die Gewichte zum einem in ihrer Summe 1 sein müssen, zum anderen können wir hier auch mit negativen Gewichten arbeiten. Am Ende erhalten wir folgende Formel:

\begin{equation}
\label{formel}
S=S0+\sum_{i}wi(Si-S0)
\end{equation}
