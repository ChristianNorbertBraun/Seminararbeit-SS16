\subsection{Skinning im Gesicht}

Wie bei den meisten ist auch bei uns das Ziel das die Betrachter auf einem emotionalen Level Bezug zu unserem Charakter aufnehmen. Daf�r ist jedoch bekannterweise das Gesicht ein wichtiger Faktor. Leider sind die verschiedenen Gesichtsausdr�cke wenig abh�ngig von unserem anatomischen Skelett. Wir k�nnen die Situation durch einsetzten von Pseudoknochen verbessern, aber nicht l�sen.

Nach /cite{dutreve2008feature} ist aber dieser interessante Ansatz entstanden der auf "{}Blend-Shaping"{} basiert. Auch wenn dieser zugegebenerma�en etwas mehr Eingriff von Seiten des K�nstlers bedarf. Die Grundlegende Idee ist einfach. Wir erstellen eine neutrale Gesichtsposition $S0$ und �berlegen uns Emotionen welche wir wollen, zum Beispiel Freude oder Stolz. F�r jede dieser Emotionen erzeugen wir ein Beispiel Gesicht ($S1$,$S2$,...). Nun erstellen wir glaubw�rdige Gesichtspositionen indem wir die Emotionen mischen und als Gewicht nehmen. Wollen wir zum Beispiel Stolz und Freude zu gleichen teilen w�re dies $0.5*S1+0.5S2$. Man beachte das bei diesem Ansatz der K�nstler grundlegend mehr Freiheiten hat und die Gewichte zum einem in ihrer Summe 1 sein m�ssen, zum anderen k�nnen wir hier auch mit negativen Gewichten arbeiten. Am Ende erhalten wir folgende Formel:

\begin{equation}
\label{formel}
S=S0+\sum_{i}wi(Si-S0)
\end{equation}
