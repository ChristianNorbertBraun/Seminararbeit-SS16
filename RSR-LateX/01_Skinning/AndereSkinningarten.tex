\subsection{Skinning im �berblick}

Es gibt verschiedene Arten von Skinning. Diese werden hier im kurzem vorgestellt. 


\subsubsection{Physiologisch basierte Ans�tze}

Dieser Ansatz basiert darauf die einzelnen Komponenten eines physischen K�rper darzustellen. Zum Beispiel durch dreidimensionale Darstellungen von Muskeln von realit�tsnahen Muskeln \cite{lee2010survey}.

\subsubsection{Aufzeichnung echter K�rperlicher Funktionen}

Dieser Ansatz basiert im wesentlichen darauf konkrete Bewegungen in einer Gruppe von Menschen aufzuzeichnen und die Resultate daraus f�r die Computer-Animation zu mitteln. Dies ist f�r konkrete physiologische Vorg�nge wie zum Beispiel Atmung sehr interessant \cite{tsoli2014breathing}. Leider zeichnen sich diese Verfahren dadurch auf das sie sehr aufw�ndig sind und oft spezialisierte Hardware ben�tigen. 

\subsubsection{Multipler Input}

Manche Techniken basieren darauf mehrere Gittergraphen bereit als Input zu verwenden um Beispielsweise eine Interpolation zu finden \cite{kavanautomatic}.

\subsubsection{Geometrischer Ansatz}

Bei diesem Ansatz haben wir ein Mesh das direkt an ein Skelett gebunden ist. Da wir in der Erstellungen unseres animierten Charakters bereits beim Rigging diesen Weg gew�hlt haben werden wir diesen nun hier fortf�hren. 