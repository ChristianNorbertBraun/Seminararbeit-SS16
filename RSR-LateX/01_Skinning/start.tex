\section{Skinning}

\subsection{Skinning}

Wie bereits etabliert beschreibt das Rigging die Bildung eines Skelettes und die Bindung eines Gittergraphen, Mesh genannt, an diesen als Repr�sentation einer Figur. Diese Technik wird in der Film-Branche weitl�ufig verwendet um den Grundstein f�r animierte Charaktere zuliefern. Der ganze Sinn des Skeletts hinter unserem Modell ist jetzt jedoch, das man durch das Skelett anatomisch glaubw�rdige Bewegungen implementieren kann. Da aber nat�rlich trotzdem das Modell selbst sp�ter das einzig Objekt im Auge des Betrachters ist muss sich dieses, ebenfalls anatomisch glaubw�rdig, mit bewegen. Dieses Vorgehen wird als Skinning bezeichnet, weil das Modell sozusagen als Haut des Skelettes mit bewegt wird. 

Bewegungen geschehen am Knochen und werden als Transformation bezeichnet. Es gibt grundlegend drei Arten von Bewegungen. Translationen beschreiben die Ver�nderung im dreidimensionalen Raum, bei gleichbleibender Ausrichtung. Rotationen beschreiben die Ver�nderung der Ausrichtung des Knochen und aller die von ihm erben, ohne eine Ver�nderung der Winkel zwischen zwei Knochen. Diese werden als feste Transformationen bezeichnet, was bedeutet das sie ihr Volumen erhalten. Die letzte Form ist keine Bewegung im eigentlichen Sinne, sondern die Ver�nderung des Gr��enverh�ltnisse der Figur, wodurch einzelne Punkte aber nat�rlich trotzdem ihre Position ver�ndern. Dies wird als Dilatation bezeichnet.

Mathematisch gesprochen wollen wir also anhand der Transformation vom Skelett eine Transformation des Gittergraphen erwirken. 

