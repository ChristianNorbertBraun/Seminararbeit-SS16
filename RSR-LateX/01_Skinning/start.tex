\section{Skinning}

\subsection{Was ist Skinning}

Wie zuvor bereits etabliert beschreibt Rigging die Bildung eines Skelettes, sowie die Bindung eines Gittergraphen, Mesh oder Modell genannt, an dieses Skelett zur Repr�sentation einer Figur. Diese Technik wird in der Filmbranche weitl�ufig als Grundstein f�r animierte Charaktere verwendet. Der Zweck des Skeletts besteht darin, anatomisch glaubw�rdige Bewegungen implementieren zu k�nnen. Da aber nicht das Skelett selbst, sondern der dar�ber gelegte Gittergraph sp�ter f�r den Betrachter sichtbar ist, muss sich dieser ebenfalls anatomisch glaubw�rdig verhalten. Dieses Vorgehen wird als Skinning bezeichnet, da das Modell sozusagen als Haut des Skelettes mit bewegt wird. 

Bewegungen geschehen am Knochen und werden als Transformation bezeichnet. F�r Skinning sind drei Arten von Bewegungen besonders wichtig: Translationen, Rotationen und Dilatationen. Translationen beschreiben die Ver�nderung im dreidimensionalen Raum bei gleichbleibender Ausrichtung. Rotationen beschreiben Ver�nderungen der Ausrichtung von Knochen und aller Objekte, die von ihm erben, ohne eine Ver�nderung der Winkel zwischen zwei Knochen. Diese werden als feste Transformationen bezeichnet, was bedeutet, dass sie ihr Volumen erhalten. Dilatationen sind keine Bewegungen im eigentlichen Sinne, sondern Ver�nderungen der Gr��enverh�ltnisse einer Figur, bei denen einzelne Punkte aber trotzdem ihre Position ver�ndern k�nnen.

Mathematisch gesprochen wollen wir also anhand der Transformation des Skelettes eine Transformation des Gittergraphen erwirken. Dabei muss jedoch beachtet werden, dass Gittergraphen sich, f�r eine realistische Darstellung, nicht immer vollkommen synchron mit ihren zugeh�rigen Skeletten bewegen sollten. Dieser Unterschied hat eine Vielzahl von Verfahren inspiriert, die im folgenden Kapitel n�her betrachtet werden sollen.

